\documentclass[footnote]{oblivoir}
% documentclass: article, book, report, amsart, amsbook, memoir, oblivoir, etc.

\usepackage{kotex, graphicx}

\title{라텍 글쓰기}
\author{김성빈}
\date{\today}

\begin{document}

\maketitle
\tableofcontents

\section{환영의 말}
안녕하세요.

Hello! World!

안녕! 세상아!

\LaTeX{}은 공백문자를      연속으로 입력해도 한     개로 인식한다.

A         B C

D  E

``졸려!''

`졸려!'

아래 문장에서 한 군데만 작게 하고 싶어요.

괜찮아요? {\tiny 작아서} 많이 놀랬죠?

\begin{footnotesize}
이 문단 전체를 작게 하고 싶어요. \footnote{이건 각주!}

작아졌지요?
\end{footnotesize}

\section{감사의 말}
감사합니다.

\begin{flushleft}
This text is\\ left-aligned. \LaTeX{} is not trying to make each line the same length.
\end{flushleft}

\begin{flushright}
This text is right-\\ aligned.
\LaTeX{} is not trying to make each line the same length.
\end{flushright}

\begin{center}
At the center \\ of the earth
\end{center}

\end{document}