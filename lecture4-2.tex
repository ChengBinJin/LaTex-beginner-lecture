\documentclass{oblivoir}

\begin{document}

유계 립쉬쯔 정역에서 정상 나비에-스토크스 방정식의 약해의 존재성은 1933년에 장 르레 \cite{J}가 증명했다. 그 후, 1934년에 장 르레는 1934년에 \cite{J2}에서 비정상 나비에-스토크스 방정식의 약해의 존재성을 증명했다. 추후에 Hopf가 1948년에 \cite{H}에 이를 명확히 진술했다. 

\begin{thebibliography}{9}
\bibitem{H} E.~Hopf, \textit{A Mathematical Example Displaying Features of Turbulence}, Comm. Pure Appl. Math. 1(1948), 303--322.

\bibitem{J}  J.~Leray, \textit{Étude de diverses équations intégrales non linéaires et de quelques problèmes que pose l'hydrodynamique}, J. Math. Pures Appl. 12 (1933), 1--82.

\bibitem{J2} J.~Leray, \textit{Sur le mouvement d'un liquide visqueux emplissant l'espace}, Acta Math. 63 (1934), no. 1, 193--248.
\end{thebibliography}


\end{document}