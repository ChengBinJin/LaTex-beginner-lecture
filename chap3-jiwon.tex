% !TEX root = Modi_RobustCode.tex

\chapter[일야구도하기]{일야구도하기(一夜九渡河記)}

\begin{flushright}
\small\sffamily 이 글은 박지원의 《열하일기》(1780년 6월부터 2개월간 겪은 일) 중 발췌하였다.\\ jiwonlipsum 패키지를 이용한다.
\end{flushright}

\section{한글 번역}
\jiwon[1-12]

\section{한글에 한자 병기}
\jiwon[13-20]

\section{원문}
\jiwon[21-28]

\section{두보의 絶句}
江碧鳥逾白 \quad  강이 푸르니 새 더욱 희고\\
강벽조유백 \\
山靑花欲然 \quad 산이 푸르니 꽃 빛이 불 붙는듯하다.\\
산청화욕연 \\
今春看又過 \quad 올봄이 또 지나가니\\
금춘간우과  \\
何日是歸年 \quad 어느 날이 이 돌아갈 해오.\\
하일시귀년 



\section{I talk to the wind}
\linespread{1.2}\selectfont
{\raggedleft King Crimson\par}
\begin{center}
Said the straight man to the late man\\
``Where have you been?''\\
I've been here and I've been there\\
And I've been in between
\medskip

(repeat) I talk to the wind\\
My words are all carried away\\
I talk to the wind\\
The wind does not hear, the wind cannot hear
\medskip

I'm on the outside looking inside\\
What do I see\\
Much confusion, disillusion\\
All around me
\medskip

I talk to the wind\\
My words are all carried away\\
I talk to the wind\\
The wind does not hear, the wind cannot hear
\medskip

You don't possess me don't impress me\\
Just upset my mind\\
Can't instruct me or conduct me\\
Just use up my time
\medskip

I talk to the wind\\
My words are all carried away\\
I talk to the wind\\
The wind does not hear, the wind cannot hear
\medskip

I talk to the wind\\
My words are all carried away\\
I talk to the wind\\
The wind does not hear, the wind cannot hear
\end{center}

\linespread{1.4}\selectfont