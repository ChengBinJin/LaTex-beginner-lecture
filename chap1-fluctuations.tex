% !TEX root = Modi_RobustCode.tex
\chapter{균형경기변동이론}

\begin{flushright}
\small\sffamily 이 글은 남상호의 《현대경제변동론》(박영사, 2003)에서 발췌하였다.
\end{flushright}


\section{서론}

앞에서 살펴본 케인지언 경기변동이론의 특징은
경기변동을 시장의 실패 혹은 경제의 불균형 상태로 인하여 발생한
것으로 파악하는데 있었다. 그런데 이들 케인즈학파의 경기변동이론은 1970년대에
발생한 스태그플레이션 현상을 예측하지 못하였을 뿐만 아니라,
사후적인 설명을 제공하는데에도 성공하지 못하였다.
그 중에서 가장 큰 문제점은 (1) 경제상황에 따라 사람들의 예측이
달라질 수 있다는 점과, (2) 경기변동모형의 핵심을 이루는 구조
파라메터의 값을 고정된 것으로 처리하였다는 점이다.

이러한 문제점을 보완하려는 노력의 일환으로 대두된 새로운 이론이
바로 균형경기변동이론\index{균형경기변동이론}(equilibrium business cycle theory\index{equilibrium business cycle theory})이다.
균형경기변동이론은 합리적 기대\index{합리적 기대}(rational expectation\index{rational expectation}),
시장청산\index{시장청산}(market clearing\index{market clearing}), 그리고 개별  경제주체의
최적화 원리\index{최적화 원리}(optimization principle\index{optimization principle})로부터 경기변동 현상이 도출된다고 본다.

개별  경제주체의 최적화 행위란 소비자는 주어진 예산제약하에서
효용을 극대화하고, 기업은 주어진 생산기술제약하에서 이윤을
극대화하는 것을 의미한다. 루카스\index{루카스}(Robert E. Lucas, Jr.\index{Lucas, R.E.})나
사전트\index{사전트}(Thomas J. Sargent\index{Sargent, T.J.}) 등은 예상하지 못한 통화량의 변동이나
외부로부터의 공급충격 등과 같은 교란요인\index{교란요인}이 발생하면 이 교란이
경제의 각 부문으로 전파되면서 개별 경제주체의 최적화 행동으로부터
경기변동이 발생한다고 본다. 결국 사람들이 미래에 대하여 합리적인
기대를 갖고, 미시경제학에 바탕을 둔 개별  경제주체의 최적화
행동원리로부터 경기변동 현상을 설명한다는 점에서 이들의 이론을
균형경기변동이론\index{균형경기변동이론}이라고 부른다.

균형경기변동이론은 경기변동의 발생원인이 무엇인가를 기준으로
화폐적 균형경기변동이론, 실물적 균형경기변동이론, 그리고 무역경기변동이론
등으로 구분된다.
화폐적 경기변동이론\index{화폐적 경기변동이론}은 예상하지 못한 통화충격이 경기변동의 원인이라고 보는 견해이고,
실물적 경기변동이론\index{경기변동이론!실물적}은 생산성 충격이나 석유파동 등과 같은 실물적 요인을
경기변동의 원인이라고 보며, 외국과의 무역으로부터 경기변동이 발생한다고 보는
견해가 무역경기변동이론이다.

다음 절에서는 화폐적 경기변동이론을 살펴보고, 3절에서는
실물적 경기변동이론을, 그리고 4절에서는 기타 경기변동이론을 개관하고자 한다.

\section{화폐적 경기변동이론}

다음과 같은 단순한 총수요--총공급 모형을 이용하여 화폐적 경기변동이론을 살펴보기로 하자:
\begin{align}
(AD) \quad y_t^d &= \beta_1 i_t + \beta_2(m_t - m_{t-1, t}^e) + \epsilon_t
\quad \beta_1, \beta_2 > 0
\\
(AS) \quad  y_t^s &= \gamma_1 i_t + \gamma_2(m_t - m_{t-1, t}^e) + \eta_t
\quad \gamma_1, \gamma_2 > 0
\end{align}
여기서 $i_t$는 $t$기의 이자율, $m_t$\은 $t$기의
통화량(자연대수값), $m_{t-1,t}^e$\은 $t-1$기에 예측한 $t$기의
통화량(자연대수값), 그리고 $\epsilon_t$와 $\eta_t$는 평균이 0이고
서로 독립인 확률적 교란항을 각각 나타낸다.

균형조건($y_t^d=y_t^s$)을 이용하여 위 식을 풀면 내생변수인 $i_t,
y_t$는 예상하지 못한 통화충격인 $m_t - m_{t-1, t}^e$ 및 확률적
교란항인 $\epsilon_t, \eta_t$의 함수로 나타낼 수 있다:
\begin{align}
y_t^* &= f(m_t - m_{t-1, t}^e, \epsilon_t, \eta_t) \\
i_t^* &= g(m_t - m_{t-1, t}^e, \epsilon_t, \eta_t)
\end{align}

결과적으로 예측하지 못한 통화공급량($m_t - m_{t-1, t}^e$)의
변화나, 수요충격\index{수요충격}($\epsilon_t$) 또는
공급충격\index{공급충격}($\eta_t$)이 균형산출량과 균형이자율에
영향을 미치게 된다. 만일 통화당국이
프리드먼\index{프리드먼}(Milton Friedman\index{Friedman, M.})의
권고에 따라 $k$\% 규칙\index{규칙}을 준수한다면 예측하지 못한
통화량의 변화는 발생하지 않으므로 통화충격으로 인한 경기변동은
발생하지 않게 된다. 그러나 현실에서는 통화당국\index{통화당국}이
준칙\index{준칙}에 입각한 통화정책을 수행하지 않거나 또는 민간
경제주체보다 우월한 정보집합을 가지고 있는 경우가 많아서 예측하지
못한 통화공급량의 변화가 존재하게 되고, 이러한 통화충격이 바로
경기변동의 원인이 되기도 한다. 다음으로 수요충격은 소비자 선호의
변화 등을 들 수 있고, 공급충격으로는 1970년대의 석유파동,
1980년대의 노동조합 파업, 1990년 및 2003년의 미국과 이라크간 전쟁
등이 여기에 속한다.

 프리드먼\index{프리드먼}\index{Friedman, M.}을 주축으로 하는
 통화주의자\index{통화주의자}들은 광범위한 실증분석을 바탕으로
 통화량의 급격한 변동이 경제를 불안정하게 만든다고 본다.
 이들이 발견한 주요 실증적 증거는 다음과 같다:
 \begin{enumerate}
 \item[(1)]
 통화공급량의 변화는 경기의 순환주기, 화폐의 유통속도,
 인플레이션 등과 강한 양($+$)의 상관관계\index{상관관계}가 존재한다.
 \item[(2)]
 통화공급량의 변화는 생산량의 변화에 선행(lead)하는 경향이 있다.
 \item[(3)]
 통화량을 주어진 것으로 보면 독립투자, 소비, 생산량 등은 유의적인
 상관관계를 갖지 못한다.
 \end{enumerate}

% 구체적으로 세계 대공황 등은
% 급격한 통화량의 감소에서 비롯된 것이라고 주장한다.

 1970년대에 들어와서 사전트\index{사전트}(T.J. Sargent\index{Sargent, T.J.})와
 노벨상 수상자인 루카스\index{루카스}(R.E. Lucas, Jr.\index{Lucas, R.E.})는 통화주의자\index{통화주의자}들의 주장을 계승\hspace*{0.25em}$\cdot$\hspace*{0.25em}발전시켜
 화폐적 균형경기변동이론을 제시하였다.

 \subsection{총수요--총공급 모형과 화폐적 균형경기변동이론}

경제의 총공급은 총생산함수로부터 구할 수 있으며,
총수요는 소비와 투자의 합으로 얻어진다.
$t$기의 총생산량은 다음과 같이 노동과 자본스톡의 함수로
나타낼 수 있다.
\begin{align}
\begin{split}
Y_t = f(L_t, K_{t-1})Y_t^s &= f\Bigl(\dfrac{P_t}{P_{t-1,t}^e}, i_{t}\Bigr) Y_t^d \\
				&= C\Bigl(\dfrac{P_t}{P_{t-1,t}^e}, i_{t}\Bigr) + I\Bigl(\dfrac{P_t}{P_{t-1,t}^e}, i_{t}\Bigr)
\end{split}
				\end{align}
여기서 ${P_t}/P_{t-1,t}^e$는 실제물가와 기대물가의
비율(상대가격)을, 그리고 $i_t$는 이자율을 각각 나타낸다.

균형조건은 다음과 같다.
 \begin{equation}
 f\Bigl(\dfrac{P_t}{P_{t-1,t}^e}, i_{t}\Bigr)
 = C\Bigl(\dfrac{P_t}{P_{t-1,t}^e}, i_{t}\Bigr) + I\Bigl(\dfrac{P_t}{P_{t-1,t}^e}, i_{t}\Bigr)
  \end{equation}
기대물가가 실제물가와 일치할 때, 즉 $P_{t-1,t}^e=P_t$가 성립할 때,
균형이 성립한다. 이 균형을 우리는 합리적
기대\index{합리적 기대}($P_{t-1,t}^e=P_t$)하의 거시경제균형\index{거시경제균형}이라고 부른다.

$P_{t-1,t}^e=P_t$가 성립하면 균형소득은 일정한 값을 갖지만 그렇지
않을 때에는 균형소득 수준은 변동한다. 구체적으로 $P_{t-1,t}^e<P_t$
또는 $P_{t-1,t}^e>P_t$가 성립하면 균형소득은 진동하게 된다.

화폐수량설\index{화폐수량설}에 의하면 통화량은 물가와 1\,:\,1
대응관계를 가지므로 예기하지 못한 통화량의 변화는 곧 기대물가와
실제물가의 괴리를 초래하게 된다. 예기하지 못한 통화량의 증가가
발생하는 경우, 실제물가 수준은 상승하지만 기대물가 수준은 이전
수준에 그대로 머물러 있게 되어 과소기대현상이 발생한다. 이렇게
되면 총공급이 총수요보다 크게 되어 이자율이 하락하고, 이자율의
하락은 소비와 투자를 증대시켜 총수요곡선을 오른쪽으로 이동시킨다.
한편 이자율의 하락은 자본의 한계생산물을 하락시켜 총공급곡선을
왼쪽으로 이동시킨다. 그런데 총수요곡선의 이동폭보다 총공급곡선의
이동폭이 더 작기 때문에 예측하지 못한 통화량의 증가는 균형소득을
증가시키는 결과를 가져오게 된다.

한편, 예측하지 못한 통화량의 감소는 실제물가와 기대물가간의
괴리($P_{t-1,t}^e>P_t$)를 발생시켜 총수요가 총공급을 초과하게
된다. 따라서 총수요곡선은 왼쪽으로 이동하고, 총공급곡선은
오른쪽으로 이동하여 궁극적으로 균형생산량이 감소하게 된다.


\begin{table}
\setlength\tabulinesep{3pt}
\caption{노동증가율의 변화와 경제변수의 성장경로}\label{tab:tab8-1}
\centering{%
\begin{tabu}{X[2.5]|X|X[2.5]|X}
\tabucline[.5pt]{-}
\rowfont{\sffamily} 경제변수             & 변화방향  &경제변수             & 변화방향   \\ 
\tabucline[.5pt]{-}
근로자당 자본($k$)&             증가 &자본($K$)&               불분명\\
근로자당 자본증가율(${\hat k}$)&불변 &자본증가율(${\hat K}$)   & 감소\\
근로자당 소득($y$)&             증가 &이자율($r$)&               하락\\
근로자당 소득증가율(${\hat y}$)&불변 &임금($w$)&                 증가\\
자본계수($v$)&             증가 &소득분배 비율($rK/wL$)&  불분명\\
국민소득($Y$)&           불분명 &근로자당  소비($C/L$)&        증가\\
국민소득증가율(${\hat Y}$)&감소 &                            &   \\
\tabucline[.5pt]{-}
\end{tabu}
}
\end{table}

