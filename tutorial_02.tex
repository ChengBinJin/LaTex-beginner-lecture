\documentclass{oblivoir}

\usepackage{mathtools}
\usepackage{amssymb, amsmath}
% amssymb: ams symbol
\usepackage{amsthm}
% amsthm: ams theorem

\newtheorem*{thm}{정리}
\newtheorem*{defn}{정의}
% * can help to ignore numbering

\begin{document}

\[ \int_{a}^b f(x)dx \]

\begin{equation}
\int_{a}^b f(x) dx
\end{equation}

$p_1$\을 재화 1의 가격이라 하고, $p_2$\을 재화 2의 가격이라 하자. 소비자가 사용할 수 있는 예산의 한도가 $m$원까지일 때, 생각할 수 있는 제약모델은 다음과 같다:
\begin{equation}
p_1 x_1 + p_2 x_2 
\leq m.
\end{equation}

자주 사용되는 효용함수로 \textbf{Cobb-Douglas} 효용함수가 있다. 이 함수의 정의는 다음과 같다:
\[ u(x_1, x_2) = x_1^c x_2^d \]

$\int_a^b f(x)dx$

\[ \int_{a}^b f(x)dx \]

\begin{equation}
\int_{a}^b f(x) dx
\end{equation}

\begin{equation}
a^x+y = a^x a^y
\end{equation}

\begin{equation*}
a^{x+y} = a^x a^y
\end{equation*}

\[ a^{x+y} = a^x a^y \]

\[ \overline{a+b} = \overline{a} + \overline{b} \]

\[ \underline{a+b} = \underline{a} + \underline{b} \]

\[ \underbrace{1+ \cdots + 1}\]

\[ \overbrace{1+ \cdots +1} \]

\[ \vec{a}=\left(3,0,0\right) \]

\[ \overrightarrow{a}=\left(3,0,0\right) \]

\[ \overleftarrow{a}=\left(3,0,0\right) \]

\[ id = \sigma^{-1} \cdot \sigma \cdot \]

\[\begin{matrix}
A & B & C \\
d & e & f \\
1 & 2 & 3 \\
\end{matrix}
\]

\[\begin{pmatrix}
A & B & C \\
d & e & f \\
1 & 2 & 3 \\
\end{pmatrix}
\]

\[\begin{bmatrix}
A & B & C \\
d & e & f \\
1 & 2 & 3
\end{bmatrix}
\]

\[ \underset{under}{baseline} \]

\[ \overset{over}{baseline} \]

\[\sum_{\substack{1 \leq i \leq  q\\
1 \leq j \leq q \\
1 \leq k \leq r}
} a_{ij} b_{jk} c_{ki}
\]

\[ A=\{ x \in \mathbb{R} | x^2=a, \text{where $a$ is positive}\} \]

\[ A=\{ x \in \mathbb{R} \mid x^2=a, \text{where $a$ is positive} \} \]

\begin{thm}
\[ A=\{ x \in \mathbb{R} \mid x^2=a, \text{where $a$ is positive} \} \]
\end{thm}

\begin{defn}
$\mathbb{R}$ is the set of all real numbers.
\end{defn}















\end{document}