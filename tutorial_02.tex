\documentclass{oblivoir}

\usepackage{mathtools}
\usepackage{amssymb, amsmath}
% amssymb: ams symbol
\usepackage{amsthm}
\usepackage{bbm}
% amsthm: ams theorem

\newtheorem{defn}{정의}
\newtheorem{add}[defn]{부연설명}
\newtheorem{thm}{정리}[section]
\theoremstyle{definition}
% * can help to ignore numbering

\begin{document}

\[ \int_{a}^b f(x)dx \]

\begin{equation}
\int_{a}^b f(x) dx
\end{equation}

$p_1$\을 재화 1의 가격이라 하고, $p_2$\을 재화 2의 가격이라 하자. 소비자가 사용할 수 있는 예산의 한도가 $m$원까지일 때, 생각할 수 있는 제약모델은 다음과 같다:
\begin{equation}
p_1 x_1 + p_2 x_2 
\leq m.
\end{equation}

자주 사용되는 효용함수로 \textbf{Cobb-Douglas} 효용함수가 있다. 이 함수의 정의는 다음과 같다:
\[ u(x_1, x_2) = x_1^c x_2^d \]

$\int_a^b f(x)dx$

\[ \int_{a}^b f(x)dx \]

\begin{equation}
\int_{a}^b f(x) dx
\end{equation}

\begin{equation}
a^x+y = a^x a^y
\end{equation}

\begin{equation*}
a^{x+y} = a^x a^y
\end{equation*}

\[ a^{x+y} = a^x a^y \]

\[ \overline{a+b} = \overline{a} + \overline{b} \]

\[ \underline{a+b} = \underline{a} + \underline{b} \]

\[ \underbrace{1+ \cdots + 1}\]

\[ \overbrace{1+ \cdots +1} \]

\[ \vec{a}=\left(3,0,0\right) \]

\[ \overrightarrow{a}=\left(3,0,0\right) \]

\[ \overleftarrow{a}=\left(3,0,0\right) \]

\[ id = \sigma^{-1} \cdot \sigma \cdot \]

\[\begin{matrix}
A & B & C \\
d & e & f \\
1 & 2 & 3 \\
\end{matrix}
\]

\[\begin{pmatrix}
A & B & C \\
d & e & f \\
1 & 2 & 3 \\
\end{pmatrix}
\]

\[\begin{bmatrix}
A & B & C \\
d & e & f \\
1 & 2 & 3
\end{bmatrix}
\]

\[ \underset{under}{baseline} \]

\[ \overset{over}{baseline} \]

\[\sum_{\substack{1 \leq i \leq  q\\
1 \leq j \leq q \\
1 \leq k \leq r}
} a_{ij} b_{jk} c_{ki}
\]

\[ A=\{ x \in \mathbb{R} | x^2=a, \text{where $a$ is positive}\} \]

\[ A=\{ x \in \mathbb{R} \mid x^2=a, \text{where $a$ is positive} \} \]

\section{Hi}
\begin{thm} \label{thm:1}
\[ A=\{ x \in \mathbb{R} \mid x^2=a, \text{where $a$ is positive} \} \]
\end{thm}
\begin{proof}[증명]
Hello world!

\[ id = \sigma^{-1} \cdot \sigma \cdot \qedhere \]
% \qedhere: put box on the top
\end{proof}

\section{Hello}
\begin{defn}
$\mathbb{R}$ is the set of all real numbers.
\end{defn}

\[ id = \sigma^{-1} \cdot \sigma \cdot \]

정리 \ref{thm:1}에 의해서

정리 \eqref{thm:1}에 의해서

Note that
\begin{equation} \label{eq:1}
A \leq B
\end{equation}
and
\begin{equation} \label{eq:2}
B \leq A.
\end{equation}
So by (\ref{eq:1}) and \eqref{eq:2}, we conclude that $A=B$.

\begin{equation}
\begin{split}
Hf(x) &= \mathrm{p.v.}\frac{1}{\pi} \int_{\mathbb{R}} \frac{f(y)}{x-y}dy\\
&= \lim_{\varepsilon \rightarrow 0} \frac{1}{\pi} \int_{|x-y|>\varepsilon} \frac{f(y)}{x-y}dy
\end{split}
\end{equation}

\begin{equation}
\begin{aligned}
Hf(x) &= \mathrm{p.v.} \frac{1}{\pi} \int_{\mathbb{R}} \frac{f(y)}{x-y}dy \\
&= \lim_{\varepsilon \rightarrow 0} \frac{1}{\pi} \int_{|x-y|>\varepsilon} \frac{f(y)}{x-y}dy
\end{aligned}
\end{equation}

\begin{align}
Hf(x) &= \mathrm{p.v.} \frac{1}{\pi} \int_{\mathbb{R}} \frac{f(y)}{x-y}dy \\
&= \lim_{\varepsilon \rightarrow 0} \frac{1}{\pi} \int_{|x-y|>\varepsilon} \frac{f(y)}{x-y}dy
\end{align}

\begin{align}
Hf(x) &= \mathrm{p.v.} \frac{1}{\pi} \int_{\mathbb{R}} \frac{f(y)}{x-y}dy \\
&= \lim_{\varepsilon \rightarrow 0} \frac{1}{\pi} \int_{|x-y|>\varepsilon} \frac{f(y)}{x-y}dy \nonumber
\end{align}

\begin{align*}
Hf(x) &= \mathrm{p.v.} \frac{1}{\pi} \int_{\mathbb{R}} \frac{f(y)}{x-y}dy \\
&= \lim_{\varepsilon \rightarrow 0} \frac{1}{\pi} \int_{|x-y|>\varepsilon} \frac{f(y)}{x-y}dy
\end{align*}

Cobb-Douglas 모델의 MRS(Marginal rate of substitution)\를 구해보도록 하자. $u(x_1, x_2) = x_1^c x_2^c$이라 할 때,
\begin{align*}
\mathrm{MRS} &= -\frac{\partial u(x_1, x)2)/\partial x_1}{\partial u (x_1, x_2) / \partial x_2}\\
&= -\frac{cx_1^{c-1} x_2^{d}}{dx_1^c x_2^{d-1}} \\
&= -\frac{cx_2}{dx_1}
\end{align*}
와 같다.

\begin{align*}
a_{11} &= b_{11} &
a_{12} &= b_{12} \\
a_{21} &= b_{21} &
a_{22} &= b_{22} + c_{22}
\end{align*}

\begin{flalign*}
a_{11} &= b_{11} &
a_{12} &= b_{12} \\
a_{21} &= b_{21} &
a_{22} &= b_{22} + c_{22}
\end{flalign*}

align환경이면서 한 행에 부연설명을 하고자 할 때 적합한 환경이다.
\begin{alignat}{2}  %영역을 크게 두 개로 나눔
x &= y_1 - y_2 + y_3 - y_5 + y_8 - \dots &\quad& \text{by } \\
&= y' \circ y^* && \text{by } \\
&= y(0) y' && \text {by Axiom 1.}
\end{alignat}

$\mathbbm{1}$

$\mathbbmss{ABCdef12}$

%$\mathbbmtt{ABCdef12}$

\end{document}