\documentclass{oblivoir}

\usepackage{mathtools}
\usepackage{amssymb, amsmath}

\begin{document}

\[ \int_{a}^b f(x)dx \]

\begin{equation}
\int_{a}^b f(x) dx
\end{equation}

$p_1$\을 재화 1의 가격이라 하고, $p_2$\을 재화 2의 가격이라 하자. 소비자가 사용할 수 있는 예산의 한도가 $m$원까지일 때, 생각할 수 있는 제약모델은 다음과 같다:
\begin{equation}
p_1 x_1 + p_2 x_2 
\leq m.
\end{equation}

자주 사용되는 효용함수로 \textbf{Cobb-Douglas} 효용함수가 있다. 이 함수의 정의는 다음과 같다:
\[ u(x_1, x_2) = x_1^c x_2^d \]

$\int_a^b f(x)dx$

\[ \int_{a}^b f(x)dx \]

\begin{equation}
\int_{a}^b f(x) dx
\end{equation}

\begin{equation}
a^x+y = a^x a^y
\end{equation}

\begin{equation*}
a^{x+y} = a^x a^y
\end{equation*}

\[ a^{x+y} = a^x a^y \]

\end{document}